\documentclass{article}

\usepackage{fontawesome}

\usepackage[utf8]{inputenc}
\usepackage[english]{babel}

\usepackage{xcolor}
\usepackage{titlesec}
\usepackage[style=ieee,defernumbers=true]{biblatex}

\usepackage{url}
\usepackage[hidelinks]{hyperref}



\definecolor{highlight}{RGB}{0,0,173}
\definecolor{blue}{rgb}{1,0.5,0}



\titleformat{\section}
  {\bigskip\color{highlight}\normalfont\Large\bfseries}{\hspace{-10pt}}{.5em}{\vspace{.5ex}\MakeUppercase}[\titlerule]
\titlespacing*{\section}{0pt}{0pt}{15pt}

\titleformat{\subsection}
  {\color{highlight}\normalfont\large\bfseries}{\hspace{-12pt}}{.5em}{\vspace{0ex}}[]
\titlespacing*{\subsection}{12pt}{0pt}{5pt}
              

%%\usepackage[hyphens]{url}
%%\hypersetup{breaklinks=true}

\let\origurl\url
\renewcommand{\url}[1]{\href{#1}{\faExternalLink}}

\addbibresource{../../vonLaszewski-jabref.bib}

\parindent0pt


\usepackage[left=0.75in,top=0.6in,right=0.75in,bottom=0.6in]{geometry}

% \usepackage{showkeys}


\begin{document}

\newcounter{varseg}
\setcounter{varseg}{0}
\newcommand{\pbib}{\stepcounter{varseg} \printbibliography[heading=none,resetnumbers=false,segment=\the\numexpr\value{varseg}]}


\begin{center}
  {\huge\bf Gregor von Laszewski, Ph.D.} \\
Research Professor, University of Virginia
\href{https://laszewski.github.io}{https://laszewski.github.io}
\url{https://laszewski.github.io}
%\\
%CV from \today~ available at % \href{https://laszewski.github.io/vonLaszewski-resume.pdf}{https://laszewski.github.io/vonLaszewski-resume.pdf}
%\url{https://laszewski.github.io/vonLaszewski-resume.pdf}\\

\end{center}

\begin{minipage}[t]{0.6\columnwidth}

Biocomplexity Institute \\
Town Center Four \\
994 Research Park Boulevard \\
Charlottesville, VA 22911
\end{minipage}
\ \
\begin{minipage}[t]{0.4\columnwidth}
\flushright
8282 South Stone Ridge Rd\\
Bloomington, IN 47401\\
(812)~824-8660\\
laszewski@gmail.com\\
\end{minipage}


%----------------------------------------------------------------------------------------
% EDUCATION SECTION
%----------------------------------------------------------------------------------------

\section{Education}

\begin{description}
\item[Ph.D., Computer Science, Syracuse University, Syracuse, NY] \hfill {\it Sep. 1991 - Nov. 1996} ~\\
      Thesis: A Parallel Data Assimilation System and Its Implications   
      on a Metacomputing Environment\\
      Advisor: Geoffrey C. Fox 

\item[Fellowship at The Ohio State University,
  Columbus, OH] \hfill {\it Sep. 1990 – Sep. 1991}~\\
  Graduate exchange student

\item[M.S. Computer Science (Diplom Informatiker), University of Bonn, Germany] \hfill {\it Sep. 1987 - Nov. 1990}~\\
  in collabooration with \\
  {\bf The German National Institute for Information Technology (GMD)}, \\
  now {\bf Fraunhofer-Gesellschaft} ~\\
  Thesis: A Parallel Genetic Algorithm for the K-way Graph Partitioning Problem  \\
  Major: Artificial Intelligence \\
  Minor: Physics \\  
  Advisor: Heinz Mühlenbein 

\item[B.S. Computer Science (Vordiplom), University, of Bonn, Germany.] \hfill {\it Sep. 1984 - Apr. 1987} ~\\  
  Major: Coputer Science\\
  Minor: Physics \\  

\end{description}
 
%----------------------------------------------------------------------------------------
% WORK EXPERIENCE SECTION
%----------------------------------------------------------------------------------------

\section{Experience}

\begin{description}

\item

\item[ Research Professor, Biocomplexity Institute] \hfill {\it Nov 2021  - present} ~\\
  University of Virginia, Charlottesville, VA

\item[ Assistant Director, Digital Science Center] \hfill {\it Jul. 2012 - Dec. 2021} ~\\
  Indiana University, Bloomington, IN

\item[ Adjunct Professor, Intelligent Systems Engineering Department] \hfill {\it Sep. 2016 - Dec. 2021} ~\\
  Indiana University, Bloomington, IN

\item[ Assistant Director of Cloud Comuting, Community GridsLab] \hfill {\it Jul. 2009 – 2015} ~\\
  Indiana University, Bloomington, IN

\item [Adjunct Professor, Computer, Science Department] \hfill {\it Sep. 2012 – Sep. 2015} ~\\
  Indiana University, Bloomington, IN

\item[	Lead Architect, FutureGrid] \hfill {\it Jul. 2009 – Jan. 2015} ~\\
  Indiana University, Bloomington, IN

\item[ 	Scientist, Argonne National Laboratory, ] \hfill {\it Apr. 2002 – Jul. 2009} ~\\
  Mathematics and Computer Science Division.  Argonne, IL. 

\item{\bf Sabbatical from Argonne National Laboratory}   \hfill {\it Aug. 2007 – Jul. 2009} ~\\
  {\bf Associate Professor, Rochester Institute of Technology, Rochester, NY} \\
  Director, Service-Oriented Cyberinfrastructure Laboratory\\
  Associate Professor, Computer Science Department\\
  Associate Professor, Ph.D. Program, GCCIS\\
  Rochester Institute of Technology, Rochester, NY

\item[	Fellow, Computation Institute, University of Chicago ] \hfill {\it Jan. 2000 – Jul. 2007} ~\\
  Computation Institute, Chicago, IL, 
  
\item[ 	Visiting Professor, University of North Texas. ] \hfill {\it Nov. 2004 - Jan. 2005} ~\\
  Department of Computer Science and Engineering, Denton, TX,
  
\item[ 	Assistant computer scientist, Argonne National Laboratory, ] \hfill {\it Nov. 1998 - Apr. 2002} ~\\
  Mathematics and Computer Science Division.  Argonne, IL.
    
\item[ 	Visiting professor/Guest Lecturer, Illinois Institute of Technology. ] \hfill {\it Jan. 2002 - Dec. 2002} ~\\
  Computer Science Department, Chigaco IL
  
\item[ 	Postdoctoral researcher,   Argonne National Laboratory ] \hfill {\it Nov. 1996 - Nov. 1998} ~\\
  Mathematics and Computer Science Division.  Argonne, IL.
  
\item[ 	Research assistant, NASA Goddard Space Flight Center] \hfill {\it Jun. 1994 - Jan. 1995} ~\\
  University Research Space Agency (USRA), Greenbelt, MD.

\item[ 	Research assistant, German National Research Center] \hfill {\it Feb. 1987 - Sept. 1990} ~\\
  {\bf  for Information Technology (GMD) } \\
  now Frauenhofer Gesellschaft, Sankt Augustin, Germany
  
\end{description}

\section{Awards}   

\begin{description}
  \item[Achievement Award.] MLCommons Science Working Group efforts
    \hfill {2022}
\item[Best Staff Award.] \hfill {\it Sept. 2018} ~\\
  School of Informatics, Computing, and Engineering, Bloomington, IN, U.S.A. 

\item [Special Judges award at the FLL Robotics World Championship] \hfill{\it 2018} ~\\
  Team coached won the special judges award,
  Detroit. The team also won multiple \\
  championship awards in the State of Indiana (2016--2018):
  % \url{https://www.firstinspires.org/about/press-room/record-crowd-inspired-by-worlds-largest-celebration-of-young-innovators-and-changemakers-at-first-championship-in-detroit}
  2016-2017 1st Place Programming Award;
  2017 B3 FLL QT Best Core Values;
  2018 B3 FLL QT Best Project award;  
  2017-2018 Southern Indiana Overall Champion’s Award;
  2018 Southern Indiana Best Presentation Award;
  2018 All Girls Champion FLL Challenge.

  
\item[Sandia National Laboratories Recognition Award]  \hfill {\it Mar. 2005} ~\\
  Member of the CMCS Team, Livermore, CA, U.S.A.
  
\item[Overall best research poster at Supercomputing 2004] \hfill {\it Nov. 2004} ~\\
  Poster won among 109 submissions. Pittsburgh, PA, U.S.A. 
  
\item[Chicago Innovation Award] \hfill {\it Oct. 2003} ~\\
  As member of the Globus Project, Chicago, IL, U.S.A.
  
\item[Department of Energy Outstanding Mentor Award] \hfill {\it Apr. 2003} ~\\
  Argonne, IL, U.S.A. \url{https://www.anl.gov/mcs/awards-recognition}

  % \url{http://www.anl.gov/Media_Center/Argonne_News/2004/an040112.html} 
  
% \item[R\&D100 Award as member of the Globus Project] \hfill {\it Oct. 2001} ~\\
%  Significant contributions as member of the Globus Project, Chicago, IL, U.S.A.
  
\item[Best of Show award in the High-Performance Computing Challenge Supercomputing‚ 98] \hfill {\it Nov. 1998} ~\\
  Received the award on stage, coordinated across multiple National Labs \\
  the application was leading to the HPC Challange award, Orlando, FL.
  
\item[Fellowship, University Space Research Agency (USRA) at Goddard Space Flight Center. ] \hfill {\it 1995}

\item[Fellowship, The Ohio State University. ] \hfill {\it 1990}

\item[Overall best student paper at Supercomputing‚ SC’92. ] \hfill {\it Oct. 1992}

\item[Department for Education and Research, Germany] \hfill {\it Sep. 1989} ~\\
  Financial waiver by due to outstanding grades upon graduation.
  
\end{description}


\subsection{Awards for Activities Supervised by Dr. von Laszewski}

\begin{enumerate}
  \item Best student research project 2015, Indiana University SOIC
  \item Supervised the building of a cluster on a wall from recycled computer parts by 8 undergraduate honors students. The resulting cluster wall was presented by a student to the Institute and he won 1 of 5 awards among 65 group presentations, Aug. 2008.
  \item  Supervised the development of a Grid certificate authority that was awarded the best student project at Polytechnic New York. 
  \item  Supervised the development of an LDAP browser that won the Novel developers award.
\end{enumerate}


%----------------------------------------------------------------------------------------
% Grants
%----------------------------------------------------------------------------------------

\section{Grants}

\begin{enumerate}

%\item{NIST}

\item AWS Cloud Award: co-PI Gregor von Laszewski, AWS Cloud Credit for Research.  Jul. 2023 - Sep. 2025, \$30K
\item Cloudbank: co-PI Gregor von Laszewski, Collaborative Research: Framework: Software: CINES: A Scalable Cyberinfrastructure for Sustained Innovation in Network Engineering and Science, Apr. 2023 - Sep. 2024, \$5300
\item NIST: PI Gregor von Laszewski, Reusable Hybrid Multi-Services Data Analytics Framework, Sep. 2021 – Aug. 2023, \$75K Total to G. von Laszewski
\item  NSF: co-PI Gregor von Laszewski, XD Metrics Service (XMS), Jul. 2015 – Aug. 2021, \#1445806 , \url{https://www.nsf.gov/awardsearch/showAward?AWD_ID=1445806}, \$750K Total to G. von Laszewski
\item  XPS: FULL: DSD: Collaborative Research: Rapid Prototyping HPC Environment for Deep Learning, July 2, 2018 - Aug 2022, Geoffrey Fox gcf@indiana.edu (PI), Judy Qiu (Co-PI), Gregor von Laszewski (Co-PI), 	\$315,000.00
\item  NSF: co-PI Gregor von Laszewski, CSR: An Analytic Approach to Quantifying Availability (AQUA) for Cloud Resource Provisioning and Allocation \#1409809/1409256, \$100K total to G. von Laszewski. 

\item  NSF: co-PI Gregor von Laszewski, XD TAS, with University of Buffalo, \$750K total to G. von Laszewski, 9/2009 – 9/2015, \url{http://www.nsf.gov/awardsearch/showAward?AWD_ID=1025159}

\item  NSF: co-PI Gregor von Laszewski, XPS: FULL: DSD: Collaborative Research: Rapid Prototyping HPC Environment for Deep Learning, \$315K total.
  \url{http://www.nsf.gov/awardsearch/showAward?AWD_ID=\#1439007}

\item  NSF: Gregor von Laszewski, Gregor von Laszewski, PI at Indiana University, Collaborative Research: DDDAS-TMRP: An Adaptive Cyberinfrastructure for Threat Management in Urban Water Distribution Systems, FY2009-2010, \$183K
  
\item  NSF: Gregor von Laszewski, OGCE, subcontract with Indiana University, \$216K 9/1/2007 – 8/31/2010
\item  Microsoft: Gregor von Laszewski, Donation of a 40 node cluster. 
\item  Microsoft: Gregor von Laszewski, PI. What to teach in advanced Cyberinfrastructure. Aug. 2008, \$5K. Summer 2008
\item  Healthcare: Gregor von Laszewski, PI at RIT and Technical Director of the Emergency Directory Service, Development of an Emergency Directory Service. Subcontract with STEP of a grant-funded through GRRHIO via the NYSDOH. Nov 2008 – Aug 2010, \$130K. Jan. 2009 – Aug. 2010. Transitioned in Aug 2009 to RIT as moving the grant was not possible.
\item  DOE: Gregor von Laszewski, PI, LDRD On-demand calculation of Advanced Photon Source Data, Awarded, Oct. 06, 2006-2007, \$300K. 
\item  DOE: Gregor von Laszewski, PI at Argonne National Laboratory SBIR on Insightful Workflows, Awarded. Sep. 06, \$30K 
\item  NSF: Gregor von Laszewski, PI at University of Chicago, Collaborative Research: DDDAS-TMRP: An Adaptive Cyberinfrastructure for Threat Management in Urban Water Distribution Systems, University of Chicago, FY2006-FY2008, \$52K, ANI0540076 
\item  DOE: Al Wagner, Branko Ruscic, and Gregor von Laszewski. Active Thermo Chemical Tables as part of the DOE SciDAC Collaboratory for Multiscale Chemical Science, \$600K, Sept. 2004 Jun. 2006. 
\item  NSF: David Angulo and Gregor von Laszewski. REU Site: An Interdisciplinary REU Site for Bioinformatics and Grids, CNS0353989, \$260K, FY2004 - FY2006. 
\item  NSF: Gregor von Laszewski. Collaborative Research: Grid Portal Middleware, ANI0330545, NMI: \$393K, 1 Sep. 2003 - Sep. 2006. 
\item  DOE: Gregor von Laszewski and Keith Jackson. Commodity Grid Kits, Enabling Middleware Gregor von Laszewski for Designing Science Applications to enable Grid workflows. \url{http://www.cogkits.org}, \$500K, Sep. - 2004 Sep. 2006. 
\item  DOE: Gregor von Laszewski and Keith Jackson. Commodity Grid Kits, Enabling Middleware for Designing Science Applications. \url{http://www.cogkits.org}, \$750K, Sep. 2002 - Sep. 2004. 
\item  NSF: Gregor von Laszewski. Java CoG Kit Technologies, NSF Grant ACI9619019, \$245K, FY2002 - FY204. 
\item  Microsoft: Gregor von Laszewski. Microsoft Equipment Donation, \$5K, FY2003. 
\item  Microsoft: Gregor von Laszewski. Microsoft Software Donation: MSDN Subscription, FY1992- FY2006. 
\item  DOE: Al Wagner, Brank Ruscic, and Gregor von Laszewski. Active Thermo Chemical Tables as part of the DOE MICS Collaboratory for Multiscale Chemical Science, \$900K, Sept. 2001 Sept. 2003. 
\item  DOE: Randy Brameley, Rick McMullan, John Hoffman, and Gregor von Laszewski. NGIA Grid-based Collaboratory for RealTime Data Acquisition IU/ANL, FWP 56890, \$300K, FY1999.

\end{enumerate} 


\section{Selected Educational Activities} 

\begin{enumerate}

\item Undergraduate Research Opportunities
  
  \begin{itemize}
    \item Fall 2022-2023, University of Virginia Biocomplexity
      Institute and Initiative, support of 2 undergarduate students
      for MLCommons benchmarks,
      \item Fall 2022-2023 support 4 undergraduate students at New
        York University for MLCommons benchmarks
    \item Summer REU 2022 University of Virginia Biocomplexity
      Institute and Initiative, 5 students supervised
  \item  Virtual Summer REU at Florida Agricultural and Mechanical University~\\
  support of 18 students from which 17 were minorities, Summer 2020
\item  Summer Research Experience at Indiana University ~\\
  Supervised 5 students from which 4 are minorities,  Summer 2021
\item  Undergraduate Research Opportunities in Computing,
  Indiana University 2020, 2021
\item Research experience for undergraduates~\\
  Leading the organization of a group of 10 undergraduate students

  % and intensely supervised 4 students. The result is published as a series of 6 posters available at
  % \url{http://cloudmesh.futuregrid.org/reu/}

% \item  AI-First, assistant in Class management, and teaching of GitHub, 2021

    \item RIT Honors Class, Parallel Compute Cluster, Rochester Institute of Technology, 2008
    \item NSF REU Site on Grid Computing and Bioinformatics, DePaul University 2004-2006 
    \item Interprofessional Project Team Course, IPRO at Illinois Institute of Technology, Chicago, IL, Spring 2003
    \item DOE/NSF Faculty-Student Teams (FaST) Program, Argonne National Labs, IL, 2003, 2002 

  \end{itemize}
  
\item  Various courses related to Grid Computing and Big Data,  Indiana University
  \begin{itemize}
        \item ENGR-E 599 Independent Study in Cloud Computing Engineering and AI, \\ Spring 2020, Fall 2020, Spring 2021
        \item CSCI-B 649 Graduate: Topics in Systems with focus on Cloud Computing, Spring 2020
        \item ENGR-E 616 Graduate: Advanced Cloud Computing, Spring 2018, Spring 2020
        \item ENGR-E 516 Graduate: Engineering Cloud Computing Spring 2018, Fall 2018, Spring 2019, Spring 2020
        \item INFO-I 423 Undergraduate: Big Data Applications and Analytics (Undergraduates), Fall 2016 
        \item INFO-I 523 Graduate: Big Data Applications and Analytics (Graduates), Fall 2016
        \item ENGR-E 599 Graduate: Topics in Intelligent Systems Engineering, Fall 2016
        \item CSCI-Y 790 Graduate: Graduate Independent Study, Fall 2014 
  \end{itemize}

  
\item  STEM student activities

  \begin{itemize}
    \item 2022-2023, K:11 Support of a high school student to be
      educated in parallel computing
    \item K5-6: Summer Camp, Indiana University, Introduction to Robotics and Python, Summer 2014
    \item K5-6: Introduction to Robot Programming, Indiana University  
    \item K-12: Introduction to Cloud Computing for K-12 students, Purdue University K-12 Education Seminar, Sept. 2012
    \item K-12: Designing a Raspberry PI Cluster Case, Indiana University
    \item K5-12: Summer Camp on Robotics, Indiana University

  \end{itemize}

\end{enumerate}

%\subsection{Supervision}
  
%Selected supervision of staff and students
%Staff:  Wang, Fugang, Abdul-Wahid, Badi, PhD
%Independent Studies: Anthony Orlowsky, Richard Otten, Arpit Agrawal, Balasubramani, Prashanth, Thanmai Bindi. TA:  Hyungro Lee. RA: Shenoy, Gourav,  Shivanand, Supreeth, Mangirish Wagle
%Volunteer: Rajagopal, Ashwini


% \subsection{Undergraduate Education Projects}

%\begin{enumerate}
%\item  REU Group at IU CGL:
  
%\item  NSF REU Site on Grid Computing and Bioinformatics, 2006, D. Angulo and Gregor von Laszewski. (8 undergrads and 1 graduate student 
%\item  NSF REU Site on Grid Computing and Bioinformatics, 2005, D. Angulo and Gregor von Laszewski. (8 undergrads and 1 graduate student) 
  %\item  NSF REU Site on Grid Computing and Bioinformatics, 2004, D. Angulo and Gregor von Laszewski. (8 undergrads and 1 graduate student) (Best poster award at SC2004)
  
% \item  Computationally Mediated Experimental Science Gregor von Laszewski, Nestor J. Zaluzec, Xian-He Sun, Interprofessional Project Team Course, IPRO at Illinois Institute of Technology, Chicago, IL, Spring 2003. (3 undergrads and 3 graduate students) \url{http://www.cogkit.org/viewcvs/viewcvs.cgi/papers/pdf/IPRO2.pdf?rev=HEAD}

% \item  DOE/NSF Faculty-Student Teams (FaST) Program, 2003 (1 faculty and 3 undergraduate students) \url{http://www.anl.gov/Media_Center/logos20-3/fasttrack.htm, http://www.scied.science.doe.gov/scied/Abstracts2003/ANLcs.htm} 

%\item  3D Visualization for NEESgrid Earthquake Engineering Laboratory Experiments. Sze Man Chan (Pace University, New York, NY 10038) Gregor von Laszewski (Argonne National Laboratory, Argonne, IL 60439). Department of Energy, Office of Science, Office of Workforce Development for Teachers and Scientists. 
%\item  3D Visualization Toolkit. Shuaib Chowdry (Pace University, NYC, NY 11103) H. Winkler (Pace University, NYC, NY 11103), 2003, Gregor fon Laszewski. 
%\item  Multithreaded Sensor Server Architecture for Instrument Monitoring. Oleg Yunakov (Pace University, New Yoork, NY 10038) 2003, Gregor von Laszewski (Argonne National Laboratory, Argonne, IL 60439). 
%\item  DOE/NSF Faculty-Student Teams (FaST) Program, Supporting Grid Computing with Java, Igor Diner, Oleg Yunakov, Dennis Anderson (Pace University, New York, NY 10038) Gregor von Laszewski (Argonne National Laboratory, Argonne, IL 60439), 2002, \url{http://www.anl.gov/Media_Center/News/2002/news020822.htm} (1 faculty and 2 undergraduate students)  
%\item  DOE SULI: A Comparative Performance Analysis of the Java CoG Kit. Mihael Hategan (Illinois Institute of Technology, Chicago, IL 60616) 2003, Gregor von Laszewski (Argonne National Laboratory, Argonne, IL 60439). 
%\end{enumerate}



\subsection{Thesis Advised}

\begin{enumerate}
\item  MS: JavaScript CoG Kit. Fugang Wang, Rochester Institute of Technology, Dec 2009. 
\item  MS: GreenIT VM scheduling. Casey Rathbone, Dec 2010.
\item  MS (Masters Project): Grid Security. Akylbek Zhumabayev, Rochester Institute of Technology, Graduation June 2009. 
\item  PhD: GridTorrents. Ali Kaplan, Indiana University, June 2009.
\item  MS: GridShell, Boris Wachtmeister, Technical University Aachen, Germany, March 2008.
\item  MS: Deployment Issues in Grid Computing. Guru Prasad, Southern Illinois University, Fall 2006. 
\item  PhD: An Integrated Architecture for Ad Hoc Grids. Kaizar Amin, Computer Science and Engineering Department, University of North Texas. Jan, 2006.
\item  BS: Grid Service Data Needed for Estimation of Reliability in Scientific Workflow Systems, Daniel Colonnese, North Carolina State University, 2004. 
\item  MS: Grid Eigen Trust, A Framework for Computing Reputation in Grids. Beulah Kurian Alunkal, Computer Science Department, Illinois Institute of Technology, Chicago. Dec 2003. 
\item  MS: The Java CoG Kit Grid Desktop, A Simple and Central Approach to Grid Computing Using the Graphical Desktop Paradigm, Pankaj R. Sahasrabudhe, University of Louisville, 2003.
\item  MS: Adapting BPEL4WS to the Grid, (committee member), Marcial Rion, Mathias Kengelbacher, A workflow language proposal for Grid environments, 2000, Main Advisor: Dr. Josef M. Joller, School of Technik, Raperswil, CH 
\item  BS: A Grid Certificate Authority, Mike Sosonkin, Polytechnic University, Brooklyn, New York (Best student project of the graduating class)
\end{enumerate}



%----------------------------------------------------------------------------------------
% HINDEX
%----------------------------------------------------------------------------------------

\section{h-index}

As of August 1, 2024 Google scholar reported 12257 and an h-index of 47


%----------------------------------------------------------------------------------------
% TECHNICAL STRENGTHS SECTION
%----------------------------------------------------------------------------------------

\section{Technical Strengths}

Throughout my carreer, I have used many software systems and
programming languages.  I have developed and overseen many different
projects that simplified utilization of Cloud, Distributed Computing,
and Grid Infrastructure. I am the lead architect of cloudmesh. I was
the lead architect of FutureGrid (a multicloud deployment) and the
Java CoG Kit which supported many application portals to provide easy
access to heterogeneous distributed HPC infrastructure including benchmarking.

\bigskip

\begin{tabular}{ll}
Cloud Computing & VM, Containers, cloudmesh interfaces with AWS, Azure, Google, Oracle\\
Web related & REST, JSON, YAML \\
Computer Languages & Python, Java, Perl, C, C++, OCCAM, and others\\
Administartion & Project execution\\
GitHub         & I am managing over 200 GitHub repositories\\
Collaboration & I am able to lead and work in large collaborative projects\\
\end{tabular}

%----------------------------------------------------------------------------------------
% STANDARDS
%----------------------------------------------------------------------------------------

\section{Standards Documents}

\begin{refsegment}

  \nocite{las-2020-nist-bigdata}
  \nocite{las-2019-nist}  
  \nocite{las-2019-nist-vol8}
  \nocite{las-2001-gosv3}
  \nocite{las-2001-gosv2}  

\end{refsegment}

\pbib

%----------------------------------------------------------------------------------------
% Copyrighted Software
%----------------------------------------------------------------------------------------

\section{Copyrighted Software}

\begin{refsegment}

\nocite{las-2000-ldap-browser}  

\end{refsegment}


\pbib



%----------------------------------------------------------------------------------------
% PROCEEDINGS
%----------------------------------------------------------------------------------------

\section{Proceedings}

\begin{refsegment}

  \nocite{las-2023-escience}
  \nocite{las-2022-reu}  
  \nocite{las-2012-fedcloud-proc}  
  \nocite{las-2011-workshop-data}
  \nocite{las-2007-gce}
  \nocite{las-2006-gce}

\end{refsegment}

\pbib


%----------------------------------------------------------------------------------------
% Class Boos and Proceedings
%----------------------------------------------------------------------------------------

\section{Selected Class Books and Proceedings}

\begin{refsegment}

  \nocite{las-2020-book-cloudeng}
  \nocite{las-2020-book-python}  
  % \nocite{las-2020-book-bigdata}
  % \nocite{las-2020-book-222}
  \nocite{las-2020-book-linux}
  \nocite{las-2020-book-markdown}
  \nocite{las-2020-book-tech}
  % \nocite{las-2020-book-chameleon}
  % \nocite{las-2020-proc-516} % missing
  % \nocite{las-proc-516-2019} % INCOMPLETE, missing epub
  \nocite{las-2018-handbook}
  \nocite{las-2018-projects-1}
  \nocite{las-2018-software-1}
  \nocite{las-2018-software-2}
  \nocite{las-2018-usecase-vol-2}
  \nocite{las-2018-usecase-vol-1}
  \nocite{las-2018-usecase-vol-3}
  \nocite{las-2018-tech-vol-8}
  \nocite{las-2018-tech-vol-9}
  \nocite{las-2018-class-proceedings}
  \nocite{las-1995-issues-parallel}
  % \nocite{las-2019-cloud-eng-lecture}


  
\end{refsegment}


\pbib

%----------------------------------------------------------------------------------------
% In Book
%----------------------------------------------------------------------------------------

\section{Book Chapters}

\begin{refsegment}


  \nocite{las-2014-bigdata}
  \nocite{las-2013-fg-bookchapter}  
  \nocite{las-2011-ondemand}
  \nocite{las-2011-greenchapter}
  \nocite{las-2009-cloudcomp}
  \nocite{las-2007-workflow}
  \nocite{las-2006-workcoordination}
  \nocite{las-2006-workflow-book}
  \nocite{las-2005-qos-book}
  \nocite{las-2004-middleware}
  \nocite{las-2004-gestalt}
  \nocite{las-2004-grid-pattern}
  \nocite{las-2004-grid-midleware}
  \nocite{las-2003-gridcomputing}
  \nocite{las-2003-cmt-book}
  \nocite{las-1998-telemedicine}
  
\end{refsegment}


\pbib

%----------------------------------------------------------------------------------------
% JOURNAL
%----------------------------------------------------------------------------------------

\section{Journal Articles}

\begin{refsegment}

  \nocite{las-2024-supercharging}
  \nocite{las-2022-crypto}
  \nocite{las-2021-covid}
  \nocite{las-2023-mlcommons-edu-eq}
  
  \nocite{las-2014-xdmod}
  \nocite{las-2013-ondemand}
  \nocite{las-2012-xdmod-kernel}
  \nocite{las-2011-energy}
  \nocite{las-2011-emolst}
  \nocite{las-2011-ann}

  \nocite{las-2010-wang-transactions}
  \nocite{las-2010-multicore}


  \nocite{las-2010-ve-journal}
  \nocite{las-2010-ijahuc}

  \nocite{las-2009-yang-advances}
  \nocite{las-2009-virt}
  \nocite{las-2009-cloud}

  \nocite{las-2007-knowlege-portal}  
  \nocite{las-2007-ogce}

  
  \nocite{las-2006-saga}
  
  \nocite{las-2006-guss-j}

    
  \nocite{las-2005-cmcs-cluster}
  \nocite{las-2005-portal-journal}
  \nocite{las-2005-qos-model}
  \nocite{las-2005-workflow-jgc}
  \nocite{las-2005-reputation-j}
  \nocite{las-2005-portalarch}
  \nocite{las-2005-gridhistory}
  \nocite{las-2005-atct}

  \nocite{las-2004-atct-j}
  \nocite{las-2004-qos-journal}
  \nocite{las-2004-ftp-journal}
  \nocite{las-2004-abstraction-j}

  \nocite{las-2003-minmin}  
  \nocite{las-2003-knowledge}

  \nocite{las-2002-perlcog}
  \nocite{las-2002-javacog}
  \nocite{las-2002-grip}
  \nocite{las-2002-corbacog}
  \nocite{las-2002-cactus-j}
  \nocite{las-2002-cactus}

  \nocite{las-2001-cog-concurency}
  \nocite{las-2001-cmt}
  \nocite{las-2001-acm}

  \nocite{las-2000-sbc}

  \nocite{las-1999-loosely}
  \nocite{las-1999-ieee}
  \nocite{las-1999-hbm-journal}


  \nocite{las-1993-lu-journal}

  
\end{refsegment}


\pbib


%----------------------------------------------------------------------------------------
% PROCEEDINGS
%----------------------------------------------------------------------------------------


\section{Article in Proceedings and Workshops}


\begin{refsegment}

  \nocite{las-2022-icbicc}
  \nocite{las-2021-openapi}
  \nocite{las-2021-impact}
  \nocite{las-2021-hptmt}
  
  \nocite{las-2020-net-science}
 

  \nocite{las-2019-streaming}
  \nocite{las-2019-harc-comet}
  
  \nocite{las-2018-impact}

  \nocite{las-2017-chameleon-mongo}
  \nocite{las-2017-chameleon-teach}
  \nocite{las-2017-teaching}
  \nocite{las-2017-comet}

  \nocite{las-2016-virtcluster}

  \nocite{las-2015-impact-ncar}
  \nocite{las-2015-impact-ncar}
  \nocite{las-2015-tas}
  \nocite{las-2015-xsede}
  
  \nocite{las-2014-Impact}
  \nocite{las-2014-multi-grid}
  
  \nocite{las-2013-gpu}
  \nocite{las-2013-xdmod}

  \nocite{las-2012-imagemanagement}
  \nocite{las-2012-comparisoncloud}
  \nocite{las-2012-xdmod-data}
  
  \nocite{las-2011-virt}  
  \nocite{las-2011-wtm-hadoop}
  \nocite{las-2011-imagerepo-a}
  \nocite{las-2011-grapple}
  \nocite{las-2011-threat}  

  \nocite{las-2010-gce}  
  \nocite{las-2010-vmschedule}
  \nocite{las-2010-onserve}
  \nocite{las-2010-energycloud}
  \nocite{las-2010-ccgrid}
  \nocite{las-2010-sla}
  \nocite{las-2010-energy}
  \nocite{las-2010-dynamic}

  \nocite{las-2009-sla}
  \nocite{las-2009-fpga}  
  \nocite{las-2009-ogce}
  \nocite{las-2009-ispan}
  \nocite{las-2009-ispa}
  \nocite{las-2009-ipccc}
  \nocite{las-2009-gridcat}
  \nocite{las-2009-cluster}
  \nocite{las-2009-ccgrid}

  % r \nocite{las-2008-msproject} % ther is something wrong
  % \nocite{las-2008-ms} % do not use
  % possibly we could not atten e-science and
  % it was published, but is no longer available
  
  \nocite{las-2008-federated-cloud}
  \nocite{las-2008-project}
  \nocite{las-2008-javascript}

  \nocite{las-2007-swift}
  \nocite{las-2007-javascript}
  \nocite{las-2007-gridtorrent}
  \nocite{las-2007-contamination}  

  \nocite{las-2006-water}
  \nocite{las-2006-cobalt}

  \nocite{las-2005-ad-hoc-sec}  
  \nocite{las-2005-exp}
  \nocite{las-2005-workflowrepo}
  \nocite{las-2005-ca}
  \nocite{las-2005-adhoc-quality}

  \nocite{las-2004-gridant}
  \nocite{las-2004-clade}
  \nocite{las-2004-adcom}
  \nocite{las-2004-abstracting}
  \nocite{las-2004-qos-ccgrid}

  \nocite{las-2003-reputation}
  \nocite{las-2003-qos}
  \nocite{las-2003-masses}
  \nocite{las-2003-ftp}
  \nocite{las-2003-dc}
  \nocite{las-2003-coalloc}
  \nocite{las-2003-chem}
  
  \nocite{las-2002-vision}
  \nocite{las-2002-pdcs}
  \nocite{las-2002-ocgsa}
  \nocite{las-2002-infogram}
  \nocite{las-2002-gcc}
  \nocite{las-2002-deploy}
  \nocite{las-2002-activetable}
  
  \nocite{las-2001-pse}
  \nocite{las-2001-hpdc-cactus}
  \nocite{las-2001-corba}
  \nocite{las-2001-greedy}  

  \nocite{las-2000-moba}
  \nocite{las-2000-grande}
  % \nocite{las-2000-mdsml} % r 
  % \nocite{las-2000-ggf-schema}  % r broken 

  \nocite{las-1999-spie}
  \nocite{las-1999-siam}
  \nocite{las-1999-rostock}
  \nocite{las-1998-hpdc}
  \nocite{las-1997-nobugs}
  \nocite{las-1997-hpdc}
  \nocite{las-1996-ecwmf}
  \nocite{las-1994-ecwmf}
  \nocite{las-1992-high}  
  \nocite{las-1991-icga}
  \nocite{las-1990-ppsn}
  \nocite{las-1990-natug}

  
\end{refsegment}


\pbib

%----------------------------------------------------------------------------------------
% THESIS
%----------------------------------------------------------------------------------------

\section{Thesis}


\begin{refsegment}

\nocite{las-1996-thesis}

\end{refsegment}

\pbib


%----------------------------------------------------------------------------------------
% POSTERS
%----------------------------------------------------------------------------------------

\section{Selected Refereed Posters at Major Conferences}

\begin{refsegment}

  \nocite{las-2023-poster-ai-for-science-nsf-nsf}
  \nocite{las-2023-poster-cylon-ornl}
  \nocite{las-2010-sc-poster-rain}

  \nocite{las-2005-scidac-poster-portal}
  
  \nocite{las-2004-scidac-poster-chem}
  \nocite{las-2004-scidac-poster-cogkit}

  \nocite{las-2000-xport}

\end{refsegment}


\pbib



%----------------------------------------------------------------------------------------
% REPORTS
%----------------------------------------------------------------------------------------

\section{Reports}

\begin{refsegment}

  \nocite{las-2024-analysispipleine}
  \nocite{madhav23-ripples}
  \nocite{las-2023-cylon}
  \nocite{las-2023-ai-workflow}
  \nocite{las-2023-hybridmulticloud}
  \nocite{las-2024-cloudmasking-b}
  \nocite{las-2023-overview-cloudmask}
  \nocite{las-2023-whitepaper-nist}
  \nocite{las-2022-cc-arxiv}
  \nocite{las-2022-crypto-arxiv}
  \nocite{las-impact-report}
  \nocite{las-2020-covid}
  \nocite{las-2020-uroc}

  \nocite{las-2017-cloudmesh}
  
  \nocite{las-2014-reu-genome}  
  \nocite{las-2014-cloud-usage}

  \nocite{las-2012-fg-1471}

  \nocite{las-2011-towards-deployment}

  \nocite{las-2008-cumulus}

  \nocite{las-2006-cmcs-report}
  \nocite{las-2006-massspec-compare}
  \nocite{las-2006-massspec-r}
  \nocite{las-2006-massspec}
  \nocite{las-2006-batch}

  \nocite{las-2005-developing}
  \nocite{las-2005-gridandt-user}
  \nocite{las-2005-gridworkflow}
  
  \nocite{las-2004-grid-portal}

  \nocite{las-2003-cog-workflow}
  \nocite{las-2002-gsfl}
  \nocite{las-2001-corba-status}
  \nocite{las-2000-grid-create}

  \nocite{las-1998-ldap}
  
  % \nocite{las-1995-infomall}
  % \nocite{las-1995-mopac}
  % \nocite{las-1995-xpvm}

  \nocite{las-1993-collection}
  \nocite{las-1993-parallelization}

  \nocite{las-1992-crpc-92260}

  % \nocite{las-80-reservation}

\end{refsegment}


\pbib


% presentations

% las-2023-c4cg,


%%%%%%%%%%%%%%%%%%%%%%%%%%%%%%%%%%%%%%%%%%%%%%%%%%%%%%%%%%%%%%%%%%%%%%

\section{Selected Invited Talks and Presentations} 

\begin{enumerate}

 \item Gregor von Laszewski, Reusable Hybrid and Multi-Cloud Analytics Service Framework,
4th International Conference on Big data, IoT, and Cloud Computing (ICBICC 2022)
\item Gregor von Laszewski, Research projects in Data Science, FAMU, Florida, Summer 2021.
\item  Data Analytics with the Big Data NIST Reference Architecture. Gregor von Laszewski, IndyPy – PyData Indy 2019. Video URL: \url{https://www.youtube.com/watch?v=-wSurPTxuXg&list=PLt4L3V8wVnF58ZFkBQur3vSWrliVI3WAr&index=6&t=0s}
\item  BigData 2017 MIDAS and SPIDAL Tutorial . Geoffrey Fox, David Crandall, Judy Qiu, Gregor Von Laszewski, Shantenu Jha, John Paden, Oliver Beckstein, Tom Cheatham, Madhav Marathe, Fusheng Wang, Bari Italy February 13-14 2017
\url{http://infomall.org/publications/SPIDALTutorialProgram-Feb2017.pdf}
\item  Cloudmesh Virtual Cluster Management for Data-Intensive Applications, Gregor von Laszewski, DePy 2015 1st Annual Conference on Python applications in Data Analysis, Machine Learning, and Web, Chicago, Ill, May 29-30 2015. Video \url{http://www.pyvideo.org/video/3539/cloudmesh-virtual-cluster-management-for-data-int}
\item  Cloudmesh, HPCS2014 Halifax, Nova Scotia, CA, June 25- 27, 2014
\item  FutureGrid, OpenCirrus Meeting, Oct 2011, Georgia Tech
\item  Panelist: Opportunities of Services Business in Cloud Age at Cloud2011, 7/8/2011 
\item  Green Computing,  IUPUI, Indianapolis, IU Energy Conference, Aug 6-7, 2009.
\item  Towards GreenIT, Indiana University, July, 2009. 
\item  Grids for Synchrotrons, ESRF Grenoble, \url{http://www.esrf.eu/}, Dec 8, 2008. 
\item  Cyberinfrastucture Research. 2008, University of Albany 
\item  Scientific Workflows. 2008, Research Computing and NYSGrid via Access Grid, Rochester Institute of Technology 
\item  Cyberinfrastucture Workflows. 2008, CCRG Rochester Institute of Technology 
\item  Cyberinfrastucture Research. 2007, IBM Raleigh 
\item  Grid Workflows, 2007, Georgia Tech
\item  Grid Workflows, 2007, University of Buffalo
\item  CoG Kits: an opportunity for Collaboration, October 2006, Southern Illinois University  
\item  The use of XML in Grids, September 2006, Loyola University Evolution of Grid Computing with education experiences, SC07 Educational Planning Workshop, July  27-30, Argonne National Laboratory. 
\item  Active Thermochemical Table Infrastructure, CMCS Workshop, Urbana, IL, June 7-8, 2006. 
\item  Scientific Process Management, NIH, Washington DC, Feb. 3, 2006. 
\item  April 2005, Georgia Tech, Java CoG Kit 
\item  Grid Workflow with the Java CoG Kit 4. TACC, Austin, TX, 13 May 2005. 
\item  Grid Programming Patterns with the Java CoG Kit 4. GlobusWorld, Boston, Massachusetts, 7-11 February 2005. 
\item  Workflow support in the Java CoG Kit 4. Boston, Massachusetts, 7-11 February 2005. 
\item  Keynote: CoG Kit Abstractions. Workshop on Grid Application Programming Interfaces in conjunction with GGF12, Brussels, Belgium, 20 September 2004. (Keynote). %\url{http://www.cs.vu.nl/ggf/apps-rg/meetings/ggf12.html}. 
\item  Java cog kit workflow abstractions. GGF Workshop Management Working Group, GGF11 The Eleventh Global Grid Forum, Honolulu, Hawaii USA, 6-10 June 2004. (Presentation). 
\item  GridAnt. GlobusWorld, San Francisco, 20 January 2004. 
\item  Grid computing. Illinois Institute of Technology, October 2004. 
\item  The State of Grid Computing in the U.S.A. In Grid Symposium of the Ministry of Science, Germany. Wissenschaftszentrum, Bonn, 28 November 2002. (Invited Talk). 
\item  The Science of Collaboratories Workshop Series The State of Collaboratory Tools and Technologies, Sponsored by the University of Michigan and the National Science Foundation. Ann Arbor, Michigan, 19-20 July 2001. (Invited Participant). 
\item  HPC Consortium Meeting, GridSIG, Sun Microsystems. Status of the Globus Project, Heidelberg, Germany, 19-20 June 2001. (Invited Talk). % \url{http://www.sun.com/productsnsolutions/edu/hpc/heidelberg.html}. 
\item  The Use of Java in High Performance Computing. In EuroPar 2000, Munich, Germany, 30 August 2000. (Panelist). 
\item  Application Programming in the Grid. Europar2000, Munich, Germany, August 28 September 1 2000. 
\item  Unicore and Globus. Unicore Meeting, Juelich, Germany, Sept 2000. 
\item  Building Portals with Java. Computing Portals Workshop, San Francisco, CA, 78 December 2000. 
\item  Gregor von Laszeswki. Using Java in Grids. In High Performance Computing and Java, number 284 in Seminar 341, Dagstuhl, Germany, 2025 August 2000. International Conference and Research Center for Computer Science. (Invited Talk) 
\item  Building Portals with CoG. Science Portals Workshop, Urbana, IL, 22-23 September 1999. 
\item  Information Services for the Common Component Architecture. Knoxville, TN, July 1999. 
\item  Application programming in the Grid. Aachen, Germany, September 1999. 
\item  Studying and Working in the U.S.A. Aachen, Germany, September 1999. 
\item  A Grid-based Computing Portal. Alliance Chemistry Portals Meeting, Urbana, IL, August 15 2000. 
\item  Using Globus and Java on Clusters and Grids. In International Workshop on Global and Cluster 
\item  Computing (WGCC2000), Tsukuba, Japan, 15-17 March 2000. (Invited Talk). 
\item  Panel on Commodity Technologies and Grid, ISCOPE 99, December 1999. (Panelist). 
\item  The Globus Grid Infrastructure. Julich, Germany, September 1999. (Invited Talk). 
\item  Dattor, Gridforum, and Computing Portals. Julich, Germany, September 1999. (Invited Talk). 
\item  Recent Development in the Globus Project. In 3rd HLRS Metacomputing Workshop, Stuttgart Germany, 6-7 June 2000. (Invited Talk). 
\item  JavaGrande Meeting at Supercomputing. Orlando, FL, November 1998. (Panelist). 
\item  Recent Development in the Globus Project. In 2nd Symposium on Multidisciplinary Environments And Applications, MAPINT 98/MDICE Workshop, Dayton, Ohio USA, August 1998. Aeronautical Systems Center (ASC), Major Shared Resource Center (MSRC), and Wright Patterson AFB. (Invited Talk). 
\item  Reusable Components of Globus-J , October 1998. Workshop of Desktop Access to Remote Resources,  
\item  JavaGrande Forum and Argonne National Laboratory, Chicago, IL.  
\item  SC98 BoF: Java Grande. \url{http://www.javagrande.org}, Orlando, FL, November 1998. (Panelist). 
\item  Gregor von Laszewski, Mary L. Westbrook, Craig Barnes, and Ian Foster. Supercomputing Data Analysis with an Example on the APS CATs. In International Workshop on New Opportunities for Better User Group Software (NOBUGS). Argonne, IL, December 1997.  
\item  The Globus Project: A Metacomputing Toolkit for Multidisciplinary Applications. In 1st Symposiumon Multidisciplinary Environments and Applications, MAPINT '97/MDICE Workshop, Dayton, Ohio USA, August 1997. Aeronautical Systems Center (ASC), Major Shared Resource Center (MSRC), and Wright Patterson AFB. (Invited Talk). 
\item  Using the Globus Metacomputing Toolkit for Seamless Computing. Supercomputing Center at ECMWF, Reading, UK, December 1997. (Invited Talk). 
\item  Introduction to Java. Illinois Institute of Technology, May 1997. 
\item  Introduction to Genetic Algorithms. Argonne National Laboratory, Summer Program, July 1998. 
\item  Parallel Optimal Interpolation. NASA Goddard Space Flight Center, June 1996. 
\end{enumerate}
 
\section{Selected Seminars and Colloquia}

\begin{enumerate}
\item  Tutorial: FutureGrid TG11, 2011
\item  Overview of FutureGrid, PTI, 2010.
\item  JavaScript CoG Kit, Teragrid 2009. 
\item  Building Commodity Grids. In CHEF Workshop. University of Michigan, Ann Arbor, 1415 October 2002. 
\item  Grid Computing: A Collaborative Approach. In Collaborative and Distance Learning Technologies (CDLT) Day at U.S. Army Engineering Research and Development Center (ERDC), Vicksburg, MS, 29 October 2002. (Invited Talk). 
\item  Gregor von Laszewski, Nestor J. Zaluzec, and Xian He Sun. Computationally Mediated Experimental Science. The Illinois Institute of Technology Inter-professional Projects Program, IPRO305, Fall 2002. A project-oriented class. 
  % \url{http://webservices.iit.edu/ipro/}
\item  Gregor von Laszewski and Xian He Sun. CS595: Grid and Ubiquitous Computing. Illinois Institute of Technology, Chicago, IL, Spring 2002. Course material, teaching, and project supervision. 
\item  Gregor von Laszewski. Java CoG Kit Tutorial at the Joint ACM Java Grande ISCOPE 2002 Conference. Seattle, Washington, 3 November 2002. 
\item  Gregor von Laszewski. The Importance of CoG Kits for Grid Users. Globus Retreat, Chicago, IL, 2001. 
\item  Gregor von Laszewski. The Java Cog Kit. Globus Retreat, Chicago, IL, 2001. 
\item  Global Grid Forum. Introduction to the Global Grid Forum Information Services Working Group, Amsterdam, The Netherlands, 2 March 2001. 
\item  IPDPS 2001. Grid Computing, Globus, and Java Interface to the Grid, San Francisco, CA, 27 April 2001. \url{http://www.ipdps.org/ipdps2001/2001_tutorial4.html}.
\item  Vladimir Getov, Jose E. Moreira, Roldan Pozo, and Gregor von Laszewski. Java for High-Performance Computing. In Java One Conference and Java Grande Conference, San Fransisco, CA, 2 June 2001. 
\item  International Symposium in High Performance and Distributed Computing. High Performance and Grid Programming in Java and Python, San Francisco, CA, 6 August 2001. Gregor von Laszewski and Steve Fitzgerald. The Globus Grid Programming Toolkit. Tutorial at SC99, Portland, OR, 1319 November 1999. 
\item  Gregor von Laszewski and Steven Fitzgerald. The Globus Grid Programming Toolkit. The 7th IEEE Symposium on High-Performance Distributed Computing, July 1998. 
\item  Introduction to the Metacomputing Toolkit. High-Performance Computing Tutorial, The National Center for Supercomputing Applications, University of Illinois at Urbana Champaign, April 1998. 
\item  Gregor von Laszewski, Mary L. Westbrook, Craig Barnes, and Ian Foster. Supercomputing Data Analysis with an Example on the APS CATs. In International Workshop on New Opportunities for Better User Group Software (NOBUGS). Argonne, IL, December 1997. % \url{http://www.aps.anl.gov/xfd/bcda/nobugs}. 
\item  Using LDAP in the real world. De Paul University, Chicago, IL, January 2000. 
\item  Gregor von Laszewski. The Java CoG Kit. Globus Retreat, Pittsburgh, PA, July 30 August 1 2000 
\end{enumerate}
 
\section{Selected Community Activities}

My community activities involve a wide variety of activities, including
program committees, grant reviews, community education, advisory
board. Most recently, however, I have refocussed the community activities while
spending my time voluntarily on educational activities of the
community. This includes extraordinary efforts with STEM students,
undergraduates and Graduate students. This effort has been most
rewarding as I have seen many students accelerating in their
educational quest and excellence manifested by several awards.

\subsection{Selected Project and Proposal Review Panel Activities} 

\begin{enumerate}
\item  STEM: Taught Programming to a team of elementary school children 5th and 6th graders that won 2nd place in the regional FLL competition 2015 (remarkable due to the ages while the winning team was high school) 
\item  Reviewer for Fonds National de la Recherche – Luxembourg, 2014 
\item  Member of the International Advisory Board of Cybera, Alberta, CA, 2008 - present.
\item  Member of the Unicore Review Committee of the German Ministry of Science, Germany. Wissenschaftszentrum, Bonn, Nov. 27 2000-2002. 
\item  Review Committee for The Engineering and Physical Sciences Research Council (EPSRC), UK. \url{http://www.epsrc.ac.uk/}. 
\item  NSF Review Panels, 2004. 
\item  DOE proposal reviews, 2002, 2003, 2004, 2005, 2006. 
\item  GCE08 Steering Committee
\end{enumerate}
 
\subsection{Chair and Steering Committee} 

\begin{enumerate}
\item  2016 IEEE International Conference on Cloud Engineering Workshop (IC2EW).
\item 1109/IC2EW.2016.66
\item  CCGrid2015, CCGrid2016, Committee Member
\item  CloudCom 2014, Track Chair
\item  ICCCT-2014, 5th International Conference on Computer and Communication Technology, Reviewer
\item  CAC 2013, Track Chair
\item  Workshop on Federated Clouds  2012 and 8th OpenCirrus Summit, (Co-Chair, Program Co-chair, Proceedings editor)
\item  Dagstuhl Seminar Mar 2009 (Co-Chair).
\item  GCE08 in conjunction with SC08, co-Chair 
\item  GCE07 Steering Committee 
\item  Grid 2007, Program vice-chair, Austin, TX collocated with Cluster 2007, Sept 17 - 21 in 2007 
\item  GCE06 in conjunction with SC06, Chair 
\item  Minisymposium on Grid Workflow, Globus World, San Francisco, 20 January 2004. Chair. 
\item  Minisymposium on Grid Workflow. Gregor von Laszewski and Ewa Delman. San Francisco, CA, 2023 January 2003. Chair. 
\item  Global Grid Forum Grid Information Services Working Group/Area Chair. 2001 - 2002. 
\item  Grid Forum Grid Information Services Working Group/Area Chair. 2000 - 2001. 
\item  GIS Working Group. GGF 2, Vienna, VA, July 1518 2001. Chair. 
\item  GIS Working Group. GGF 1, Amsterdam, NL, March 29 2001. Chair. 
\item  Joint ACM Java Grande ISCOPE 2001 Conference. Stanford University, California, June 24 2001. Committee and Tutorials Chair. % \url{http://www.inria.fr/JGI2001}
\item  Second Workshop on Desktop Access to Remote Resources. Sandia National Laboratory, Albuquerque, NM, 15-16 February 1999. Steering Committee. 
\item  First Workshop Desktop Access to Remote Resources, 89th October 1998. Steering Committee and Chair. 
\item  Computing Portals Workshop and 3rd International Workshop on Desktop Access to Remote Resources. Together with Computing Portals Working Group (Datorr), Java Grande Forum, and The 3rd International Symposium on Computing in Object-Oriented Parallel Environments, San Francisco, California, U.S.A.,  7 December 1999. Steering Committee and Chair. %\url{http://www.acl.lanl.gov/iscope99/}
\item  IEEE Task Force on Cluster Computing. 1998-2000. Advisory Committee. 
\item  SC98 BoF: Desktop Access to Remote Resources.  Gregor von Laszewski. Orlando, FL, November 1998. 
\item  Organizer of the BoF. Supercomputing  SC93. Portland, OR, Nov. 15-19 1993. Best Student Paper Selection Committee. 
\end{enumerate}
 
\subsection{Selected Committee Review Activities}

\begin{enumerate}
\item  Committee Co-chair CloudCom2012 in the topic Cloud Computing on HPC, 2014.  
\item  4th Workshop on Scientific Cloud Computing (ScienceCloud) 2013 June 17th, 2013, New York City, NY, USA (Committee).
\item  IEEE CloudCom 2012, 2011, 2010 (Committee).
\item  ITSM2012 (Committee)
\item  PPAM2011 (Committee).
\item  TerraGrid 2010 (Awards Committee)
\item  FGMMS2010 (Committee).
\item  CCGrid 2010, 2008, 2007, 2006, 2005 (Committee) in conjunction with SC’.
\item  GCE, 2010, 2009, 2005 (Committee).
\item  ICPADS, 2010 (Committee).
\item  IGCC 2010 (Committee).
\item  IPDPS 2009, 2008 (Committee).
\item  Grid 2009, 2007 (Committee).
\item  Cluster 2009 
\item  TeraGrid 2009 (Student Mentor).
\item  GCE08 (Committee and co-Chair).
\item  Euro-Par 2008 (Committee).
\item  ICPADS'2007. 13th International Conference on Parallel and Distributed Systems, December 5 - 7, 2007 at National Tsing Hua University , Hsinchu , Taiwan. (Committee). 
\item High-Performance Computing Symposium (HPC 2007) Norfolk, VA (Committee). 
\item  CCGrid2007 (Committee). 
\item  HPC 2007: High Performance Computing Symposium, \url{http://hosting.cs.vt.edu/hpc2007/}, March 25-29, 2007 (Committee) 
\item  2-nd IEEE International Conference on e-Science and Grid Computing, Dec. 4- 6, 2006, Amsterdam, Netherlands, (Committee). % \url{http://www.escience-meeting.org/eScience2006/} 
\item  IEEE 2006 International Symposium on Modern Computing, 3-6 October 2006, Sofia, Bulgaria, (Committee) 
\item  Grid2006, (Committee) 
\item  ICCGI06 International Multi-Conference on Computing in the Global Information Technology, March 15, August 1-3, 2006, Bucharest, Romania, (Advisory Committee) 
\item  Supercomputing, SC06, SC03, SC00, SC98, SC97, SC95.
\item  PPAM 2005, Sixth International Conference on Parallel Processing and Applied Mathematics.
  % \url{http://ppam.pcz.pl/call.htm}. 
\item  IPDPS, 19th IEEE International Parallel and Distributed Processing Symposium. Denver, Colorado, 48 April 2005. 
\item  EuroPar 2006, 2005 
\item  Coordination Abstractions for Worldwide Computing, Software Technology Mini-Track, HICSS-38, Jan. 2005. % \url{http://www.hicss.hawaii.edu/}
\item  Third International Symposium on Automated Technology for Verification and Analysis, Taipei, Taiwan, 4-7 October 2005. 
\item  International Conference on eScience and Grid Technologies 2005. Melbourne, Australia, 5-8 December 2005. 
%  \url{http://www.gridbus.org/escience}
\item  6th IEEE/ACM International Workshop on Grid Computing (Grid 2005) held in conjunction with SuperComputing 2004. Seattle, WA,Nov. 2005.  %\url{http://pat.jpl.nasa.gov/public/grid2005/} 
\item  Challenges of Large Applications in Distributed Environments (CLADE). Honolulu, HI, 7 June 2004. 
%  \url{http://www.caip.rutgers.edu/clade2004/}
\item  Workshop on Component Models and Systems for Grid Applications, held in conjunction with ICS 2004: 18th Annual ACM International Conference on Supercomputing, Saint-Malo, France, June 26-July 1, 2004. 
\item  EuroPar 2004. Pisa, Italy, 31 Aug.3rd Sept. 2004. % \url{http://www.di.unipi.it/europar04/} 
\item  BioGrid'04, Second International Workshop on Biomedical Computations on the Grid, held in conjunction with, 4-th IEEE/ACM International Symposium on Cluster Computing and the Grid, Chicago, Illinois, USA, April 19-22, 2004 
\item  Workshop on Component Models and Systems for Grid Applications, Held in conjunction with ICS 2004, 18th Annual ACM International Conference on Supercomputing. SaintMalo, France, June 26 July 1 2004. 
\item  5th IEEE/ACM International Workshop on Grid Computing (Grid 2004) held in conjunction with SuperComputing 2004. Pittsburgh, USA, 8 November 2004.  % \url{http://www.gridbus.org/grid2004/} 
\item  Advanced Computing and Communications ADCOM2004. Ahmedabad Gujarat, India, 15-18 December 2004. 
\item  IPDPS 2003, International Parallel and Distributed Processing Symposium (IPDPS). Nice, France, 26 April 2003. 
\item  International Conference on Machine Learning and Cybernetics 2003. % \url{http://www.icmlc2003.hbu.edu.cn}. 
\item  The Eleventh International Conference on Parallel Architectures and Compilation Techniques. Charlottesville, Virginia, September 22-25 2002. 
\item  SAINT2002. The 2002 Symposium on Applications and the Internet. Nara City, Nara, Japan, 28 Jan -1 Feb 2002. % \url{http://www.icse.eecs.uic.edu/saint2002/} 
\item  The 2002 Symposium on Applications and the Internet. Nara City, Nara, Japan, 28 Jan 1 Feb 2002. %\url{http://www.icse.eecs.uic.edu/saint2002/}
\item  Joint ACM Java Grande and ISCOPE 2002 Conference. Seattle, Washington, November 3-5 2002. % \url{http://charm.cs.uiuc.edu/javagrandeIscope/} 
\item  CCGrid 2002, 2nd IEEE/ACM International Symposium on Cluster Computing and the Grid. Berlin, Germany, 21-24 May 2002. % \url{http://ccgrid2002.zib.de/}
\item  1st IEEE/ACM International Symposium on Cluster Computing and the Grid. Brisbane, Australia, 15-18 May 2001. 
\item  EuroPar 2001. Manchester, UK, http://europar.man.ac.uk/, 2831 August 2001. 
\item  IPDPS 2001, International Parallel and Distributed Processing Symposium . San Fransisco, CA, 27 April 2001. \url{http://www.ipdps.org/ipdps2001/} 
\item  Second International Workshop on Infrastructure for Agents, MAS, and Scalable MAS. The 5th International Conference on Autonomous Agents, Montreal, Canada, May 28 - June 1 2001. \url{http://www.cs.cf.ac.uk/User/O.F.Rana/agents2001/}
\item  Java in High-Performance Computing at HPCN Europe 2001 Conference. Amsterdam, The Netherlands, 
\item  ACM Java Grande 2000 Conference. San Francisco, California, June 3-5 2000. 
\item  International Workshop on Metacomputing Systems and Applications. Toronto, August 2124  in conjunction with the 29th International Conference on Parallel Processing, August 21 2000. 
\item  Ninth SIAM Meeting on Parallel Processing. Minisymposium on Innovative Wide Area Applications, San Antonio, TX, 2224 March 1999. Session Host. 
\item  International Conference on Parallel and Distributed Processing Techniques and Applications. Las Vegas, Nevada, USA, June 30 - July 2 1999. 
\item  International Telemedical Information Society (ITIS) Symposium, 1998. 
\item  PDPTA 1998. Las Vegas, Nevada, August 1998. Session Host. 
\item  Supercomputing SC93. Portland, OR, Nov. 15-19 1993. 
\item  Java for High-Performance Networking, 1997, 1998, 1999, 2000, 2001. 
\end{enumerate}
 
\subsection{Journal Article Reviews}

\begin{enumerate}
\item  Journal for Grid Computing. 
\item  Concurrency and Computation: Practice and Experience. 
\item  IEEE Concurrency. 
\item  Parallel Computing 
\item  Computing in Science and Engineering
\item  Journal of International Telemedical Information Society (ITIS) Letters. 
\end{enumerate}


\section{Software Developed}

\subsection{Selected Current Software and Online Projects}

\begin{refsegment}
  
  \nocite{las-2021-cybertraining-pub}  
  \nocite{las-2020-github-bookmanager}
  \nocite{las-2020-github-cloudmesh-community}
  \nocite{las-2020-github-cloudmesh}
  \nocite{las-2015-portal-futuresystems}
  
\end{refsegment}
Defining schemas for the grid information services
\pbib

\begin{enumerate}
\item  More than 70 projects at Cloudmesh Code Repositories, 2021-04, \url{https://github.com/cloudmesh}
\item  More than 174 projects at Cloudmesh Community 2021-04, \url{https://github.com/cloudmesh-community}
\item  Cloudmesh for SDSC virtual cluster project (Architect of client)
\item  FutureGrid Software (Architect)
\item  FutureGrid Image Management (Architect)
\item  FutureGrid Cloud Accounting (Architect)
\item  TAS  XDMoD (accounting report generator,  software science impact)
\end{enumerate}


\subsection{Selected Completed projects} 

AdHoc Workflows; Grid Desktop   GridFTP user interface; GridRLS user interface; GridAnt; Grid Certificate Authority; Qstat Monitor for Cobalt and PBS queuing systems; GSFL – Grid Service Flow Language; Cyberaide; Gridshell; Grid MS Project; Java CoG Kit; GridScript; GridWorkflow (Karajan); GridTorrent; Adaptive workflows for threat management analysis; Real-time analysis of advanced photon source data; Coordination of fast nuclear reactor simulations; Developed a sophisticated Metacomputing environment allowing seamless uses of multiple supercomputers; Developed one of the very first parallel Genetic algorithms for k-way Graph Partitioning. 


% #######################################################################################
\end{document}
% #######################################################################################


